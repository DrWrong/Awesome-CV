%-------------------------------------------------------------------------------
%	SECTION TITLE
%-------------------------------------------------------------------------------
\cvsection{其他经历}


%-------------------------------------------------------------------------------
%	CONTENT
%-------------------------------------------------------------------------------
\begin{cventries}

%---------------------------------------------------------
  \cventry
    {个人项目} % Affiliation/role
    {\href{https://github.com/DrWrong/monica}{Monica}} % Organization/group
    {} % Location
    {2016年3月} % Date(s)
    {
      \begin{cvitems} % Description(s) of experience/contributions/knowledge
      \item {一个旨在提高开发效率, 适用于RESTFUL的golang框架}
      \item {项目在夺宝吧开发中得到了很好的运用很大程度的提高了生产效率}
      \end{cvitems}
    }

%---------------------------------------------------------
  \cventry
    {个人项目} % Affiliation/role
    {\href{https://github.com/DrWrong/code_snap}{CodeSnap}} % Organization/group
    {} % Location
    {2015年5月} % Date(s)
    {
      \begin{cvitems} % Description(s) of experience/contributions/knowledge
        \item {效率工具: 解决在开发过程中创建新项目需要复制粘贴的痛点, 一条命令搞定项目创建}
      \end{cvitems}
    }

%---------------------------------------------------------
  \cventry
    {核心成员} % Affiliation/role
    {ISCC(北京理工大学信息安全对抗赛) 组委会} % Organization/group
    {北京} % Location
    {2012.10 - 2015.6} % Date(s)
    {
      \begin{cvitems} % Description(s) of experience/contributions/knowledge
      \item {伴随着整个大学经历, 对网络安全 二进制安全均有比较深入的了解与学习}
        \item {基于Vmware开发出线下赛环境自动部署系统,相当于是一次运维开发的尝试,该系统后来被openstack取代}
      \end{cvitems}
    }

%---------------------------------------------------------

%---------------------------------------------------------
%---------------------------------------------------------
\end{cventries}
